\documentclass[11pt]{article}

\usepackage[utf8]{inputenc}
\usepackage[T1]{fontenc}

\usepackage[a4paper, left=2cm, right=2cm, top=3.5cm, bottom=3.5cm]{geometry}
\usepackage[french]{babel}

% Paragraph spacing
\setlength{\parskip}{1em}

% Fancy headers
\usepackage{fancyhdr}

% Captions for subfigures
\usepackage{subcaption}

% Code highlighting
\usepackage{minted}

% Footnote inside a caption
\usepackage{fnpos}
\usepackage{ftnxtra}

% Maths
\usepackage{amsmath}
\usepackage{amssymb}

% Todo notes
\usepackage{todonotes}

% Table of contents for bibliography
\usepackage[nottoc]{tocbibind}

% Inline monospace font
\def\code#1{\texttt{#1}}

% Figures
\usepackage{graphicx}

% Draw figures
\usepackage{tikz}

% Tikz node rotation
\usetikzlibrary{positioning}

% Turing machine
\usetikzlibrary{chains,fit,shapes}

% Usage: \rotnode[options]{rotation}{text}
\newcommand\rotnode[3][]{%
    \node [#1, opacity=0.0] (tmp) {#3};
    \node [draw, rotate around={#2:(tmp.center)}] at (tmp) {#3};
}

% Clickable links
\usepackage{hyperref}

% Table of contents depth
\setcounter{tocdepth}{2}

% Inline code
\usepackage{listings}
\usepackage{color}

\title{Droit : cas pratique}

\author{William SCHMITT}
\date{2018-2019}

\begin{document}
\maketitle

\section{Phase d'embauche}
\subsection*{Question 1 : recrutement}
Offre d'embauche : entente sur presque tout, sauf la rémunération
Promesse d'embauche : entente sur tous les éléments, promesse unilatérale

Dans ce cas, il n'y a pas d'accord sur la rémunération et l'entreprise ne s'est pas engagée. Il ne s'agit donc pas d'une promesse.

S'il y a accord sur les éléments importants du contrat, cela peut constituer une promesse.

\subsection*{Question 2 : CDD répétés}
Sur 2 ans, on peut penser que l'augmentation de l'activité n'est pas temporaire, et que le contrat devrait être requalifié en CDI et obtenir une indemnité (souvent égale à 1 mois de salaire).

\section{Exécution du contrat}
\subsection*{Question 3 et 4 : période d'essai}
La période d'essai ne peut pas être reconduite \textbf{automatiquement}. En cas d'arrêt maladie, la période d'essai est prolongée de la durée du dit arrêt. 
Le caractère discriminatoire peut être lié à une éventuelle absence de fautes.

L'annulation de la formation semble jouer en sa faveur.

\subsection*{Question 5 : prime}
Droit à la prime : non

\subsection*{Question 6 : non-concurrence}
Non, car il pourrait encore signer des contrats dans une entreprise d'un même secteur d'activité.

\subsection*{Question 11 : }
Question ouverte : si caractère disciplinaire : pas de préavis, sinon pas d'indemnité puisqu'incapacité à effectuer le préavis.

\subsection*{Question 12 :}
Plus de liberté (négociation indemnité + date), et plus facile pour le moral

\end{document}